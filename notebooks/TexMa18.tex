\documentclass[11pt,a4paper,ngerman]{article}
\usepackage[utf8]{inputenc}
\usepackage[T1]{fontenc}
\usepackage{amsmath,amsfonts,amssymb,graphicx,listings,color,float,wrapfig,mathtools,longtable,caption,pxfonts}
\usepackage[autostyle]{csquotes}
\usepackage{textcomp,gensymb}
\usepackage{multicol}
\usepackage[left=3cm, right=3cm, top=4cm]{geometry}
\usepackage{scrextend}
\usepackage{letltxmacro}

\newcommand*{\mypcr}{\fontfamily{pcr}\selectfont}
\DeclareTextFontCommand{\textCode}{\mypcr}

% Colors

\definecolor{dkgreen}{rgb}{0,0.7,0}
\definecolor{gray}{rgb}{0.5,0.5,0.5}
\definecolor{mauve}{rgb}{0.58,0,0.82}
\definecolor{lightgray}{rgb}{.9,.9,.9}
\definecolor{darkgray}{rgb}{.4,.4,.4}
\definecolor{orange}{rgb}{0.8, 0.4, 0}
\definecolor{ylgreen}{rgb}{0.3, 0.7, 0}
\definecolor{purple}{rgb}{0.7, 0, 0.7}
\definecolor{dkred}{rgb}{0.8, 0, 0}

% Define JavaScripts
\lstdefinelanguage{JavaScript}{
  keywords={typeof, new, true, false, catch, function, return, null, catch, switch, var, if, in, while, do, else, case, break, let, const, for},
  keywordstyle = {\color{purple} \bfseries},
  keywordstyle = {[2]\color{dkred}},
  keywordstyle = {[3]\color{black} \bfseries},
  commentstyle = {\color{darkgray}},
  stringstyle = {\color{ylgreen}},
  numberstyle = {\color{orange}},
  ndkeywordstyle = {\color{orange}\bfseries},
  basicstyle = {\color{black} \ttfamily},
  keywords = [2]{(, ), {, }},
  keywords = [3]{==, +, =, +=, -=, --, ++, ;, !},
  ndkeywords = {class, export, boolean, throw, implements, import, this},
  sensitive = false,
  comment = [l]{//},
  morecomment = [s]{/*}{*/},
  morestring = [b]',
  morestring = [b]"
}

\author{Finn Harbeke}
\title{Künstliche Intelligenz}
\date{17. November, 2018}

\begin{document}
\setlength\parindent{0pt}
\maketitle
\tableofcontents

\section{Intro}

\subsection{KI}

\subsection{Meine Projekte}

\section{Snake-Teil}

\subsection{Idee und Ziel}

Das Ziel bei dem ersten Teil meiner Arbeit ist es, verschiedene Arten von Evolutionen zu vergleichen. Diese Evolutionen sollen möglichst effizient  Künstliche Intelligenzen erschaffen, die das Game \enquote{Snake} erfolgreich spielen. Bei einer computer-simulierten Evolution lernt das Individuum, also eine bestimmte Künstliche Intelligenz, nicht selbst, sondern es reagiert auf eine Situation immer gleich. Der eigentliche Prozess des Lernens geschieht über mehrere Generationen, wie wir es von der evolutionären Anpassung\footnote{https://de.wikipedia.org/wiki/Evolutionäre\_Anpassung \textit{aufgerufen am...}} kennen. Es gibt eine Gruppe von Schlangen mit DNA, die das Game spielen und dabei einen möglichst hohen Score anstreben. Im nächsten Schritt wird die Gruppe anhand ihrer Leistungen selektioniert. Die erfolgreichen Schlangen generieren Nachwuchs, die \enquote{gute} DNA von ihren Eltern erben.

\bigskip
Ausschlaggebend sind dabei drei Faktoren. Die Selektion, das \enquote{Crossover}, die Art des DNA-Mischens, und die Mutation. Diese liegen nicht im Fokus der Arbeit. Ich wende diese Faktoren an, um beim Vergleich der folgenden zwei Evolutionsprinzipien gute Bedingungen zu schaffen.

\bigskip
Das eine dieser zwei Evolutionsmechanismen, die ich vergleiche, ist von der Natur inspiriert und die andere ist die in Machine Learning herkömmliche Methode.

\bigskip
Um diese Evolutionen zu simulieren, benutze ich Genetische Algorithmen, die beliebteste Technik von evolutionären Algorithmen, die ich im folgenden Kapitel erkläre. Danach zeige ich die Unterschiede zwischen den Evolutionen auf und stelle eine Hypothese auf, wie der Vergleich ausgehen wird. Folgend kreiere ich eine Versuchsanordnung, in welcher ich diesen Vergleich optimal durchführen kann. Zum Schluss evaluiere ich die Ergebnisse des Vergleichs, um die Hypothese entweder zu bestätigen oder zu widerlegen. 

\subsection{Genetischer Algorithmus}

In diesem Kapitel erkläre ich die Mechanismen eines Genetischen Algorithmus. Dieser ist die Methode, die ich benutzen werde, um eine Künstliche Intelligenz zu trainieren, die das Game \enquote{Snake} spielen wird. Der Genetische Algorithmus ist die Grundlage, um die Vorgänge der Evolutionen, die ich simuliere, zu verstehen.

\bigskip
Ein genetischer Algorithmus, meist Genetic Algorithm genannt, besteht aus drei essentiellen Prozessen der Evolution, welche an einer Population\footnote{Gruppe von Individuen einer Spezie, die untereinander die Möglichkeit zu Fortpflanzung haben. https://de.wikipedia.org/wiki/Population\_(Biologie), aufgerufen am 21.10.2018} ausgeübt werden. Diese sind \enquote{natürliche Selektion}, \enquote{Crossover} und \enquote{Mutation}.

\bigskip
In der natürlichen Selektion muss das Ziel eines genetischen Algorithmus definiert werden, d.h. es wird ein Fitness-Score verteilt, der die Leistung eines Individuums inb der gestellten Aufgabe darstellt. Dann wird mit den besten Individuen eine nächste Generation gezüchtet. In diesem Schritt wird der grösste Unterschied zur realen Evolution ersichtlich; während sich dort eine Population konstant fortpflanzt und Nachwuchs kriegt, wird beim normalen Genetischen Algorithmus eine Generation in einem einzigen Moment durch eine neue ersetzt. In der uns bekannten natürlichen Selektion ist es so, dass die, die länger überleben, sich tendenziell eher fortpflanzen und somit das bessere Genmaterial häufiger wird. Beim Genetischen Algorithmus ist die Wahrscheinlichkeit, Nachwuchs zu zeugen, schlicht jeweils proportional zur Fitness eines Individuums. Somit hat einer, der zehnmal so gut war wie ein anderer, auch eine zehnmal so grosse Wahrscheinlichkeit, bei der Fortpflanzung mitzuwirken. Jedes Mal, wenn diese im Genetischen Algorithmus stattfindet, wird die ganze Elterngeneration mit der neuen Generation ausgewechselt. Folglisch muss die Funktion der natürlichen Selektion, die solche Fitness-Scores berechnet, wirklich auf das gesuchte Ziel hinsteuern, was je nach Aufgabenstellung verschieden schwierig ist\footnote{Stellen Sie sich vor man müsse das Wesen eines einzelnen Menschen auf einen Fitness-Score reduzieren.}.

\bigskip
Die natürliche Selektion ist deshalb einer der heikelsten Faktoren der Versuchsanordnung, da sie den grössten Einfluss haben kann, bzw. ihr Einfluss am meisten zwischen positiv und negativ schwanken kann.

\bigskip
Das Crossover ist der Prozess der Bildung einer \enquote{Kinder}-DNA anhand zweier \enquote{Eltern}-DNAs. Zum einen gibt es die Möglichkeit beim DNA-Strang eine, zwei oder mehr Schnittstellen zu wählen und jeweils die Informationen links einer Schnittstelle vom einen, die rechts vom anderen Elternteil zu nehmen. Diese Schnittstellen sind an zufälligen Punkten im Strang. Das sieht so aus:

\begin{center}
\includegraphics[width=0.4\textwidth]{Downloads/OnePointCrossover.png}
\includegraphics[width=0.4\textwidth]{Downloads/TwoPointCrossover.png}\footnote{https://en.wikipedia.org/wiki/Crossover\_(genetic\_algorithm) aufgerufen am 21.10.2018}
\end{center}

Die andere verbreitete Methode, die, welche ich verwende, wird \enquote{uniform crossover} genannt. Bei ihr wird bei jedem einzelnen Gen entschieden, ob das des Vaters oder das der Mutter ausgewählt wird. Dabei ist die Wahrscheinlichkeit in den meisten Fällen je 50\%.

\bigskip
Das Crossover ist zwar grundlegend, welche Art benutzt wird, ist jedoch nicht so entscheidend, was das Endresultat angeht.

\bigskip
Der letzte Schritt des Genetischen Algorithmus ist die Mutation, das bedeutet, bei jedem neuen Nachwuchs gibt es eine gewisse Wahrscheinlichkeit, dass Gene zufällig mutieren. Durch Mutation kann aber ständig neue Erbinformation geschaffen werden. Denn \enquote{Mutieren} heisst entweder ein Gen etwas zu verändern, oder mit einem ganz neuen Wert zu ersetzen. Die Wahrscheinlichkeit, mit der dieser Prozess stattfindet, wird \enquote{mutation rate} genannt. Wenn sie zu klein ist, wird kaum neues Genmaterial geschaffen und die Population wird schnell zu einer kleinen Diversität streben. Zudem kann es möglicherweise sein, dass in der ersten Generation das gesuchte Genmaterial schlicht nicht vorhanden ist, dann nützt noch so viel Crossover und natürliche Selektion nichts, um an diese zu gelangen. Nur die Mutation kann das erreichen.

\bigskip
Es ist auch nicht wünschenswert, dass die Mutationsrate zu gross ist, weil dann alle Fortschritte der natürlichen Selektion und des Crossovers durch all die Mutationen immer wieder zunichte gemacht werden. Trotz guter geerbter DNA wird die Kinder-DNA nämlich so stark mutiert, dass sie nichts mehr taugt.

\bigskip
Folglich ist es bei der Durchführung eines Genetischen Algorithmus wichtig, vorallem die natürliche Selektion und die Mutationsrate im Gleichgewicht zu halten. Eine zu exponentielle Fitness-Funktion der natürlichen Selektion, die gute Individuen zu stark bevorteilt, würde das in einer Population vorhandene Genmaterial sehr schnell dezimieren, bis alle Individuen beinahe die gleiche Erbinformation haben. Eine kleine Mutationsrate würde diesen Prozess sogar noch unterstützen.

\bigskip
Im anderen Extrem wäre die Fitness-Funktion sehr flach und der Unterschied des Fitness-Scores eines schlechten und eines guten Individuums wäre so gering, dass die natürliche Selektion sozusagen nicht stattfinden würde. Oder, wie gesagt, es könnte eine zu grosse Mutationsrate den Fortschritt zwischen zwei Generationen jedes Mal rückgängig machen.

\bigskip
Mit diesen und anderen Faktoren werde ich im Kapitel \ref{sec:FaktorenVergleich}, Vergleich der verschiedenen Faktoren, experimentieren, um eine möglichst gute Versuchsanordnung für den Vergleich der Evolutionsmechanismen zu schaffen.

\subsubsection{Beispiel Shakespeare-Affen}

Mit diesem Beispiel veranschauliche ich die Prozesse eines Genetischen Algorithmus, damit die vorhin behandelte Theorie sich festigt. Es handelt sich um ein Szenario, dessen Lösung keinen Nutzen hat, aber bestens den Zweck eines Beispielszenarios erfüllt.

\bigskip
\begin{wrapfigure}[20]{l}{0.3\textwidth} 
    \vspace{-20pt}
        \begin{center}
            \includegraphics[width=0.2\textwidth]{Downloads/gaprocess.png}
            \caption{genetischer Algorithmus}
        \end{center}
    \vspace{-20pt}
\end{wrapfigure} 

Bei diesem Beispiel gibt es ein ganz klares Ziel: Schreibe den Satz \enquote{to be or not to be that is the question}. Das Bild ist eines von ganz vielen Affen, die auf Schreibmaschinen rumtippen. Die Frage ist, wie lange es dauert, bis ein Affe zufälligerweise den Shakespeare-Satz schreibt. Um einen Satz, der 39 Zeichen lang ist zu schreiben, gibt es $ 27^{39} = 6.656* 10^{55}$ Möglichkeiten. Denn bei jedem Zeichen gibt es, 26 Möglichkeiten für die Buchstaben (keine Gross- \& Kleinschreibung) und zusätzich eine für einen Abstand. Das heisst es ginge ewig, wenn die Affen immer wieder zufällig 39 Zeichen eintippten. Doch mit einem Genetischen Algorithmus wird der Prozess um ein Vielfaches verkürzt. Man initialisiert eine Population von beispielsweise 1000 Shakespeare-Affen, ihre DNA sind 39 Zeichen lange Listen, von zufälligen Buchstaben. In diesem Beispiel gibt es keine Performance, die ihr Können anders als die DNA widerspiegelt. Man könnte sagen, die Performance ist das Tippen auf der Schreibmaschine. Die Regel ist, dass ein einzelner Affe jedes Mal das Gleiche schreibt, wie es in seiner DNA geschrieben steht. Dann folgt die natürliche Selektion, bzw. die Verteilung von Fitness-Scores an alle Affen. Hier kann man die Anzahl richtiger Buchstaben nehmen.

\bigskip
Beispiel:

\bigskip
\begin{addmargin}[2em]{0em}
\begin{center}
\textCode{AffeXY:\space\space"ocxjgnfx dzjoscjfgcvizdtplgwro twtodmnf"\\
Lösung: "to be or not to be that is the question"\\
\setlength\parindent{110pt}\indent$\uparrow$\space\space\space\space\space\space\space\space\space\space\space\space\space\space\space\space\space\space\space\space\space$\uparrow$\space\space\space\space\space\space\space\space\space\space}\\
\end{center}
\end{addmargin}

\setlength\parindent{0pt}
Der Fitness-Score ist 2, da an der neunten und and der neuntletzten Stelle ein Abstand ist. Alle anderen Zeichen sind falsch. Es folgt die eigentliche natürliche Selektion, bei der jeweils zwei Eltern ausgewählt werden, um ein Kind zu zeugen, bis die neue Generation die gleiche Grösse hat wie die alte. Die Wahrscheinlichkeit, das ein Affe hierfür ausgewählt wird, ist proportional zu seinem Score.

\bigskip
Mathematisch sieht das folgendermassen aus:

\bigskip
\begin{addmargin}[2em]{0em}
\begin{center}
$ a = $ Liste aller Affen,\\
$ a_x = $ ein Affe aus der Liste an Stelle $x$ (0-basiert),\\
$ S_{a_x} = $ Score des Affens $ a_x $ und \\
$ P_{a_x} = $ Wahrscheinlichkeit des Affens $ a_x $, für die Fortpflanzung ausgewählt zu werden

\bigskip
\Large$ P_{a_x} = \frac{S_{a_x}}{\sum_{i=0}^{999} S_{a_i}} $\\
\end{center}
\end{addmargin}

\normalsize
In Python:

\lstset{
  language=Python,
  aboveskip=3mm,
  belowskip=3mm,
  showstringspaces=true,
  columns=flexible,
  basicstyle={\ttfamily},
  numbers=none,
  numberstyle=\tiny\color{gray},
  keywordstyle=\color{blue},
  commentstyle=\color{dkgreen},
  stringstyle=\color{mauve}
}

\begin{addmargin}[2em]{0em}
\begin{lstlisting}
def probability(x):
    s = sum([monkey.fitness for monkey in monkeys])
    return monkeys[x]/s
\end{lstlisting}
\end{addmargin}

Bei jedem ausgewählten Elternpaar folgt das Crossover. Wichtig! Es gibt keine Geschlechter, um welche man sich kümmern muss.

\bigskip
Es wird \enquote{uniform crossover} benutzt:

\bigskip
\begin{addmargin}[2em]{0em}
\begin{center}
\textCode{parent1: "\textcolor{red}{rbnfemtfqgrbe\space\space ygdhuxywvyjvqecizhhfwwdz}"\\
parent2: "\textcolor{dkgreen}{ankixhhognzqxfverrh fvq caqgnidbrdy izo}"\\
child:\space\space"\color{dkgreen}ank\textcolor{red}{f}xhh\textcolor{red}{fq}nzqxf\space\textcolor{red}{y}rrh\textcolor{red}{uxy}q \textcolor{red}{y}a\textcolor{red}{vq}nidb\textcolor{red}{hh}y\textcolor{red}{ww}zo\color{black}"}\\
\end{center}
\end{addmargin}

\bigskip
Zum Schluss wird bei allen neu entstandenen neuen Affen mutiert. Dieser Prozess, der in der echten Welt etwas grausam tönt, ist hier harmlos. Bei einer Wahrscheinlichkeit von beispielsweise 1\% wird ein Buchstabe in der DNA zufällig ausgewechselt, d.h. in unserem Beispiel würde es durchschnittlich sogar nur bei 2 von 5 Affen ein Gen mutiert werden.

\bigskip
\begin{center}
\textCode{child before mutation: "ankfxhhfqnzqxf yrrhuxyq yavqnidbhhywwzo"\\
          child after mutation: \ "ankfxhhfqnzqxf yrrhuxyq \textcolor{red}{p}avqnidbhhywwzo"\\}
\end{center}

Somit entsteht eine neue Generation, der wieder verschiedene Fitness-Scores zugeteilt werden und die dann eine 3. Generation erzeugt. So geht es weiter und weiter. Dieses Szenario kann beendet werden, sobald die Lösung gefunden worden ist, da es eine eindeutige Lösung gibt, in anderen Aufgabenstellungen fährt man mit dem Genetischen Algorithmus fort, bis zufriedenstellende Resultate erscheinen.

\subsection{Natur-Evolution und Computer-Evolution}

Im Folgenden werde ich die beiden Evolutionen, welche ich am Ende dieses Teils meiner Maturaarbeit vergleiche, genau definieren. Es ist insofern ebenso wie das vorige ein sehr wichtiges Kapitel, um das Ganze zu verstehen.

\bigskip
Die Begriffe Natur- und Computer-Evolution sind keine allgemein bekannte Fachbegriffe, sondern zwei von mir gewählte Ausdrücke. In der Folge verwende ich diese beiden Ausdrücke für zwei verschiedene Mechanismen der Evolution.

\bigskip
Wie das Wort \enquote{Natur-Evolution} impliziert, habe ich diese Methode von Vorgängen unserer Natur abgeschaut und werde sie anhand der menschlichen Fortpflanzung erklären. Der Mensch hat 46 Chromosomen, 23 von der Mutter und 23 vom Vater, das heisst er hat jedes Chromosom doppelt. Bei jedem einzelnen Gen setzt sich das stärkere durch, es ist das dominante. Unabhängig davon, ob bei einem Chromosom mehrheitlich die Gene des Vaters oder der Mutter benutzt wurden, ist die Chance beider Chromosome aller 23 Paare gleich gross, ans Kind weitergegeben zu werden.

\bigskip
Bei der \enquote{Computer-Evolution} hingegen wird bei der \enquote{Zeugung} jedes Gen entweder von der Mutter oder vom Vaters vererbt, d.h. sie arbeitete bei uns Menschen nur mit 23 Chromosomen, die jeweils aus Genen beider Elternteile bestehen. Somit verlieren die Chromosome ihre Bedeutung.

\bigskip
Im Folgenden erkläre ich diese Prinzipien im Detail mit bildlichen Darstellungen.

\pagebreak
\subsubsection{Beispiele}

Anhand der folgenden Beispiele zeige ich die Unterschiede der Systeme der Natur- und der Computer-Evolution detailliert auf.\\
\begin{figure}[H]
    \begin{center}
        \includegraphics[width=0.9\textwidth]{Downloads/naturdiaTry2.png}
        \caption{Natur-Evolution} \label{fig:natur}
    \end{center}
\end{figure}

Dies ist ein Beispiel von Crossover, das gemäss der Natur-Evolution funktioniert, mit Lebewesen, die ein Chromosomenpaar mit je zehn Genen haben. Bei der Natur-Evolution sieht man, dass jedes Individuum zwei Exemplare von dem Chromosom hat, eines des Vaters und eines der Mutter. Es vererbt dann eines dieser zwei Chromosome seinem Kind. Bei diesem wird dann bei jedem Gen das dominante bestimmt. Hier ist jeweils die grössere Zahl das dominante, in der Natur wäre es das, welches mehr Proteine, z.B. Pigmente, herstellt.

\bigskip
Egal, ob bei einem Chromosom mehr dominante Gene vorhanden sind, wie zum Beispiel wie das von seiner Mutter (Grossmutter der Kinder) geerbte Chromosom (Grün), haben beide die gleiche Wahrscheinlichkeit weitervererbt zu werden. Das heisst, alle der vier möglichen Kinder sind exakt gleich wahrscheinlich, gezeugt zu werden.

\bigskip
Bei dieser Architektur, also wie die DNA aufgebaut ist, sind für das Kind jeweils 4 Möglichkeiten von Zweierpaaren möglich, für jedes Chromosom der Mutter zwei des Vaters. Aber bei jedem Gen können höchstens drei verschiedene Gene wirklich in Gebrauch kommen, also dominant werden. Denn beim ersten Gen wird die \enquote{1} nie dominant, also zwei der Kinder haben es in ihrem Erbmaterial aber gebrauchen es nicht. Trotzdem hat dieses Gen die gleiche Wahrscheinlichkeit, wie die, die dominant wurden, an die dritte Generation weitergegeben zu werden.\\
\begin{figure}[H]
    \begin{center}
        \includegraphics[width=0.9\textwidth]{Downloads/compdiaTry2.png}
        \caption{Computer-Evolution} \label{fig:comp}       
    \end{center}
\end{figure}

Die Computer-Evolution funktioniert simpler. Jedes Individuum hat immer nur eine Version eines Gens in der eigenen Erbinformation gespeichert, so gibt es jeweils nur eine Auswahl zwischen zwei Genen. Diese erfolgt zufällig, das heisst, ob nun das Kind beim ersten Gen das des Vaters mit dem Wert \enquote{8} oder die \enquote{7} der Mutter erbt, ist gleich wahrscheinlich. Es sind hier nicht alle möglichen Kinder aufgelistet, sondern einfach vier Beispiele.

\bigskip
\subsubsection{Folgerung}


\renewcommand{\arraystretch}{1.5}
\begin{longtable}{p{0.25\textwidth-2\tabcolsep} p{0.35\textwidth-2\tabcolsep} | p{0.35\textwidth-2\tabcolsep}}
    \captionsetup{width=1\textwidth}
    \caption{Unterschiede zwischen Natur-Evolution und Computer-Evolution} \\

    \multicolumn{1}{c}{} & \multicolumn{1}{c|}{\textbf{Natur-Evolution}} & \multicolumn{1}{c}{\textbf{Computer-Evolution}} \\ \cline{2-3} \\
    \endhead
    
    \cline{2-3} \multicolumn{1}{c}{} & \multicolumn{1}{c}{} & \multicolumn{1}{r}{{nächste Seite}} \\
    \endfoot

    \cline{2-3} \\
    \endlastfoot
    
    \underbar{Doppelt versus} \underbar{einfach} & Eine DNA beinhält jeweils zwei Exemplare pro Chromosom. Das führt zu mehr verschiedenen Genen, die vererbt werden können und somit grösserer Diversität. & Jedes Gen existiert genau einmal pro DNA. Somit gibt es genau zwei Werte, die ein Gen des Kindes haben kann. \\ \cline{2-3}
    
    \underbar{Dominanz} & Da jedes Gen doppelt vorhanden ist, muss ausgewählt werden, welches dominant ist, also benutzt wird. Folglich kann auch ein anderes Gen vererbt werden, als man selbst benutzte. & Keine Dominanz nötig. Es wird immer das Gen vererbt, welches auch benutzt wurde.\\ \cline{2-3}
    
    \underbar{Kombinationsmög-} \underbar{lichkeiten / mögliche} \underbar{Kinder} & Pro Chromosom gibt es vier Möglichkeiten. Die Anzahl möglicher Kombinationen beim Crossover ist folglich abhängig von der Anzahl Chromosomen. & Für jedes Gen zwei Möglichkeiten. Die Anzahl Kombinationsmöglichkeiten ist abhängig von der Anzahl Gene.
    
\end{longtable}
\renewcommand{\arraystretch}{1}
\vspace{-2em}

Etwas mehr zu den Kombinationsmöglichkeiten: Wie man in Abbildung~\ref{fig:natur} sieht, sind genau 4 Kinder möglich. Doch es gibt auch die Möglichkeit, dass eine DNA aus mehreren Chromosomen besteht, was in der Natur auch die Norm ist. Wenn ein bestimmtes Lebewesen zwei Chromosome hat, kann man sich vorstellen, dass es für das erste Chromosom 4 Möglichkeiten gibt (siehe Abbildung~\ref{fig:natur}) und für das zweite genau gleich. Die Erbinformation eines Kindes besteht dann aus einem der vier möglichen Chromosomenpaare für das erste und einem der vier möglichen Chromosomenpaare des zweiten. Somit gibt es $4^2=16$ mögliche Kombinationen, bzw. Kinder. Daraus folgt, dass diese ANzahl Kombinationsmöglichkeiten bei der Natur-Evolution von der Anzahl Chromosome abhängt.

\bigskip
Bei der Computer-Evolution ist das nicht so. Es gibt genau gleich viele mögliche Kinder eines Elternpaares, wenn eine DNA aus zwei Chromosomen mit je vier Genen besteht, wie aus einem Chromosom mit acht Genen. Insofern ist eigentlich der Begriff der Chromosome für die Computer-Evolution schlicht unnötig. Denn jedes einzelne Gen wird schliesslich zufällig entweder vom Vater oder der Mutter geerbt. Völlig egal in welchem Chromosom es sich befindet, von welchem Elternteil das vorige Gen geerbt wurde oder was die eigentlichen Werte der zwei zur Auswahl stehenden Gene sind.

\bigskip
Die Kombinationsmöglichkeiten dieser beiden Mechanismen können mit den folgenden zwei Formeln ausgedrückt werden:

\bigskip
$k=\#Kombinationen$\\
$c=\#Chromosome$\\
$g=\#Gene$

\bigskip
\large
$k_{Natur-Evolution}=4^c$

\bigskip
$k_{Computer-Evolution}=2^g$

\bigskip
\normalsize
Folglich gibt es bei uns Menschen $4^{23}$ verschiedene Kombinationen, die ein einzelnes Paar zeugen kann, da wir zur Natur-Evolution gehören und 23 Chromosomenpaare haben. Zu beachten ist dabei, dass sehr viele unserer Gene (fast) gleich sind und somit Überschneidungen vorkommen. Zudem wird die sichtbare Anzahl nochmals verkleinert, weil bei jeder Kombination jeweils das stärkere Gen benutzt wird und somit die dominante, also erkennbare Erbinformation weniger divers sein kann, als die geerbte DNA. Jedoch werden diese Informationen vom Moment der Zeugung an von äusseren Faktoren beeinflusst, was zu (beinahe) unendlicher Diversität führt. Bei simulierter Evolution ist das nicht der Fall.

\subsubsection{Hypothese}

Wenn es mehr als doppelt so viele Gene wie Chromosome hat, was üblich ist, gibt es bei der Computer-Evolution weit mehr, und somit auch mehr erfolgreiche, Kombinationsmöglichkeiten. Daraus schliesse ich, dass die Computer-Evolution sehr schnell Fortschritte machen wird. Doch die Natur-Evolution hat ebenfalls ihre Vorteile, erstens ist doppelt so viel Erbinformation im Spiel, wenn die Gruppengrösse die gleiche ist und zweitens werden langfristig mehr verschiedene Gene im Umlauf sein. Ich denke, die Diversität spielt in der Evolution eine entscheidende Rolle. Deshalb stelle ich folgende Hypothese auf: Die durchschnittliche Fitness aller Schlangen der Natur-Evolution wird nach mehreren Generationen die der Computer-Evolution einholen und überholen. 

Dies wäre eine befriedigende Bekräftigung der Evolutionsprinzipien unserer Natur. \textit{Dazu ist zu ergänzen, dass die Mechanismen der Computer-Evolution in der Natur gar nicht funktionieren würden, da die DNA eines Kindes nicht aus zwei Eltern-DNAs zusammengestellt werden kann wie mit einem Baukasten. Denn Spermium wie auch Eizelle müssen jeweils eine vollständige DNA beinhalten. Würden sie, unwissend welche Bauteile der Andere mitbringt, nur einzelne Bausteine mitbringen, könnten Leerstellen und ungewollte zweifache Ausführungen eines Elements vorkommen.}

\subsection{Neuronale Netzwerke} \label{sec:NN}

Im folgenden Kapitel werde ich Neuronale Netzwerke erklären. Sie sind das Gehirn einer jeden Schlange in meinem Genetischen Algorithmus. Die DNA einer Schlange beinhaltet die Information, die den Denkprozess des Gehirns einer Schlange bestimmt.

\bigskip
Es ist \textit{für den Leser} nicht entscheidend, im Folgenden zu verstehen, wie Neuronale Netzwerke funktionieren, da die Endresultate auch ohne dieses Wissen verständlich werden.

\bigskip
Neuronale Netzwerke sind grundlegende Systeme in der Welt von Machine Learning. Es ist eine Methode, um mit einer Anzahl Inputs (Liste von Zahlen) zu rechnen und daraus eine Anzahl Outputs zu generieren. Ein typisches Beispiel ist, anhand von Länge und Breite von Blütenblättern die Blumensorte zu bestimmen\footnote{https://en.wikipedia.org/wiki/Iris\_flower\_data\_set}. Ein Netzwerk mit den richtigen Werten, um anhand dieser Informationen die Blumensorte zu bestimmen ist eine Künstliche Intelligenz.

\bigskip
Ein Neuronales Netzwerk besteht jeweils aus mehreren Layers, die alle aus einigen Neuronen bestehen. Die Neuronen einer Layer sind durch Synapsen mit den Neuronen der nächsten Layer verbunden. Dieses System ist von der Architektur unseres Gehirns inspiriert, allerdings funktioniert unser Gehirn noch einiges komplexer.

\bigskip
Ersichtlicher wird das mit zwei Beispielen von Neuronalen Netzwerken, welche Gleichheit erkennen sollten. Sie haben zwei Inputs, je eine 1 oder eine 0 und haben die Aufgabe, 1 ausrechnen, wenn beide Inputs gleich sind und 0 wenn sie verschieden sind. Zuerst zeige ich ein Beispiel eines sehr gut funktionierenden Netzwerks dann eines mit zufällig initialistierten Werten. Anhand dieser beiden Beispiele werde ich die inneren Prozesse eines Neuronalen Netzwerkes erklären. 

\begin{figure}[h]
    \begin{center}
        \includegraphics[width=1\textwidth]{Downloads/goodANND.png}
        \caption{funktionierendes Gleichheits-NN}
    \end{center}
\end{figure}

Jedes Input-Neuron hat eine Verbindung zu allen Neuronen der nächsten Layer, die "hidden layer", diese sind mit einem Gewicht versehen. Das Zwischen Resultat eines Neurons der Hidden Layer ist die Summe aller Inputs jeweils mit ihrem Gewicht multipliziert plus die "bias" des Neurons, die oben links zu sehen ist. An diesem Resultat wird noch eine sogenannte "activation function" angewendtet. Oftmals ist das eine Funktion wie Sigmoid oder aber eine nicht stetige wie "Rectified Linear Units" oder ReLU. Hier wird ReLU benutzt, welche eigentlich einfach negative Zahlen in nullen umwandelt und positive so lässt. Activation-Functions sind sehr wichtig da sie die Linearität der Rechnungen beseitigen, ohne sie wären alles nur lineare Operationen (Multiplikation und Addition), doch ohne Linearität können komplexere Verhältnisse ausgedrückt werden. Die Resultate der Hidden-Layer werden dann wieder mit Gewichten multipliziert und zu einer Bias addiert. Das geht noch einmal durch die ReLU-Funktion und schon hat man den Output des neuronalen Netzwerkes.

\begin{figure}[h]
    \begin{center}
        \includegraphics[width=1\textwidth]{Downloads/randomANND.png}
    \caption{Zufällig initialisiertes, nicht funktionierendes Gleichheits-NN}
    \end{center}    
\end{figure}



Das ist der Rechenweg für das Zwischenresultat des oberen Hidden-Layer-Neuron des zufällig initialisierten Netzwerks beim Inputs-Vektor $\left(\begin{array}{cc} 1 & 1\end{array}\right) $:

\bigskip
    $ ReLU(1 * -0.0053 + 1 * 0.6943 + 0.4179) = 1.1069 $

\bigskip
    Doch für das ganze Netzwerk schreibt man lieber in Matrix-Multiplikation:

\bigskip
$ ReLU(input * weights_{ih} + bias_h) = hiddenOutput $

\bigskip
$ReLU(\left(\begin{array}{cc} 1 & 1\end{array}\right) * \left(\begin{array}{cc} -0.0053 & 0.4379\\ 0.6943 & 0.9846\end{array}\right) + \left(\begin{array}{cc} 0.4179 & -0.6687\end{array}\right)) = \left(\begin{array}{cc} 1.1122 & 0.3159\end{array}\right)$

\bigskip
    $ ReLU(hiddenOutput * weights_{ho} + bias_o) = Output $

\bigskip
$ReLU(\left(\begin{array}{cc} 1.1122 & 0.3159\end{array}\right) * \left(\begin{array}{cc} 1.0212\\ 0.8301\end{array}\right) + \left(\begin{array}{cc} -0.5795\end{array}\right)) = \left(\begin{array}{cc}  1.1766\end{array}\right)$

\bigskip
Sehr spannend ist es ausserdem, wenn man den Denkmechanismus solcher Netzwerke genauer anschaut. Meistens ist es äusserst komplex und nicht zu verstehen, doch beim *and*-Problem kann man die einzelnen Schritte erkennen. Wenn man sich das funktionierende Netzwerk anschaut, sieht man, das obere Hidden-Neuron gibt nur einen positiven Output, wenn der Input $ \left(\begin{array}{cc} 1 & 1\end{array}\right) $ ist. Das heisst, es hält sozusagen Ausschau nach dem 1 1 Muster. Das untere Hidden-Neuron reagiert analog nur auf den $ \left(\begin{array}{cc} 0 & 0\end{array}\right) $ Input, sonst ist es 0. Der Outputput zeigt dann lediglich an, ob eins der Hidden-Neuronen ein Signal gibt. So simpel ist das *and*-Problem gelöst.

\bigskip
    Neuronale Netzwerke sind für Machine Learning deshalb so wichtig, weil sehr komplexe Verhältnisse zwischen Input-Vektoren und Output-Vektoren verkörpern können. Denn die Anzahl der Layers wie auch die Anzahl Neuronen in den Layers können unglaubliche Dimensionen annehmen. Das passiert dann vorallem im Bereich von Machine Learning, der Deep Learning genannt, der heisst so, weil die Netzwerke so "tief" sind, d.h. unzählige grosse Hidden-Layers haben. Deep Neural Networks, werden zum Beispiel bei der Bildererkennung benutzt. Doch zu beachten ist, dass Neuronale Netzwerke allein noch nichts mit dem Prozess des Lernens zu tun haben. Sie sind lediglich komplizierte Funktionen. Um zu lernen, können dann verschiedene Techniken benutzt werden, die Gewichte und die Biases gezielt verändern.

\subsection{Kreation meines Projektes}

Dieses Kapitel geht darum wie ich mein Snake-Projekt aufgebaut und so weit programmiert habe, dass es für den schlussendlichen Vergleich bereit war. \textit{Es ist das einzige, in welchem ich etwas genauer auf meinen Code eingehe. Zuerst erläutere ich einzelne Teile der Programmierung des eigentlichen Games, dann schauen wir uns das Neuronale Netzwerk der Schlangen an, welche selbst spielen, d.h. die, die dann Bestandteil des genetischen Algorithmus sein werden. Anschliessend, zeige ich, wie einige der Funktionen des genetischen Algorithmus in JavaScript, einer webbasierten Programmiersprache aussehen. Zum Schluss analysiere ich noch verschiedene Einstellungen, wie die Mutationsrate und verschiedene Arten der natürlichen Selektion, um das Projekt für den Vergleich der Natur- und der Computer-Evolution mäglichst schön vorzubereiten.}

\subsubsection{Snake}

Das Spiel Snake\footnote{https://de.wikipedia.org/wiki/Snake\_(Computerspiel)} ist ein Klassiker der frühen Computerspiele. Viele kennen das Spiel wahrscheinlich schon, doch ich werde trotzdem noch einmal die Regeln klarstellen, um zu zeigen, was in meiner Version gilt, da es viele Versionen gibt.

Snake ist ein zweidimensionales Game, das aus einem Rechteck aus vielen kleinen Feldern besteht. Die äussersten Zeilen und Spalten dieser Felder sind Wände, die das Spielfeld begrenzen. Der Spieler steuert eine Schlange, die sich horizontal und vertikal durch die Welt bewegen kann. Dabei ist das Ziel durch das Kreuzen von zufällig auftauchenden Essensfeldern, bzw. das Essen von zufällig auftauchenden Äpfeln zu wachsen. Denn mit jedem gegessenem Apfel, ich nenne es ab jetzt einfach Frucht, wächst die Schlange um eine bestimmte Anzahl Felder. Gleichzeitig darf sie nicht in Wände oder den eigenen Schwanz fahren, sonst ist \enquote{Game Over}. Manchmal gibt es ausserdem noch eine zweite Futterart, dessen Aufnahme ebenfalls zu \enquote{Game Over} führt.

\bigskip
\textbf{Regeln:}

\begin{itemize}
    \item Das Spielfeld besteht aus $40 * 40$ Feldern.
    \item Die Schlange kann sich nur hoch, runter, nach rechts oder links bewegen.
    \item Am Anfang ist die Schlange 4 Felder lang.
    \item Sie beginnt in der Mitte Spielfelds.
    \item Das Auftreffen des Kopfes der Schlange mit einem Wand-Feld oder dem Schwanz führt zu \enquote{Game Over}.
    \item In jedem Moment ist genau eine Frucht im Spielfeld.
    \item \underbar{Speziell:}
    \begin{itemize}
        \item Sodass es nicht zu schnell zu schwierig wird, wächst die Schlange mit jeder gegessenen Frucht nur ein Feld. Sobald sie länger als zehn Felder ist, jedoch vier.
        \item Um Unendlich-Schlaufen zu verhindern, verhungern die Schlangen nach einer zu langen Zeit ohne Aufnahme einer Frucht.
    \end{itemize}
\end{itemize}

\subsubsection{Struktur des neuronalen Netzwerks}

Dieses Unterkapitel zeigt, wie das neuronale Netzwerk der Schlangen funktioniert. Dafür muss man das Kapitel \ref{sec:NN} Neuronale Netzwerke nicht unbedingt verstanden haben, da es vorallem darum geht, anhand welcher Informationen eine Schlange ihre Entscheidungen trifft.

Die SmartSnake besitzt als DNA die Werte eines Neuronalen Netzwerks. Dieses Netzwerk hat die Struktur (24, 16, 4), d.h. sie hat 24 Inputs, eine Hidden-Layer mit 16 Neuronen und 4 Outputs. Die vier Outputs sind Werte, die jeweils den Drang der Schlange, in eine bestimmte Richtung zu gehen, numerisch darstellen. Anders gesagt gibt das Netzwerk vier Zahlen aus, die sagen, wie fest eine Schlange in jede Richtung gehen möchte. Und zwar in folgender Reihen folge: $ \left(\begin{array}{cccc} hoch & rechts & runter & links\end{array}\right) $. Wäre der Output also $ \left(\begin{array}{cccc} 1 & 0 & -1 & 2\end{array}\right) $, würde die Schlange somit ausrechnen, am liebsten in die letzte Richtung, also nach links, zu gehen. Die neuronalen Netzwerke rechnen, wie gesagt, anhand 24 Informationen aus, in welche Richtung zu gehen. Diese 24 Informationen bestehen aus 3 Achtern-Sets, jedes Achter-Set besteht aus acht Richtungen, in welche die Schlange sehen kann, nämlich alle 45\degree (360/60). Sie sieht Wandfelder, Schlangenfelder und Früchte. Diese drei "Sichtarten" entsprechen den Achter-Sets. Bei der Wand- und der Schlangensicht ist der Input jeweils 1 / Distanz zum gesehenen Feld, sodass die nächsten Wand- und Schlangenfelder die grössten Werte abgeben. Bei der Fruchtsicht ist der Input immer 0, ausser eine Frucht ist in Sicht, dann ist er 1. Die Reihenfolge dieser acht Informationen jeweils von Richtung zwölf Uhr im Uhrzeigersinn herum bis nach oben links.

\begin{center}
\begin{figure}[H]
    \includegraphics[width=0.325\textwidth]{Downloads/wallsight.png}
    \includegraphics[width=0.325\textwidth]{Downloads/snakesight.png}
    \includegraphics[width=0.325\textwidth]{Downloads/fruitsight.png}
    \caption{Wand-, Schlangen- und Frucht-Sicht}
    \vspace{-2em}
\end{figure}
\end{center}

\begin{multicols}{3}
    Bei dieser Situation ist der vierte Input der Wand-Sicht der grösste, weil die Wand nach unten am nächsten ist. Es folgt rechts (dritter Input), da die rechte Wand nur ein Feld weiter entfernt ist als die untere. Unten links und unten rechts haben den gleichen Wert.
    
    \columnbreak
    
    Bei der Schlangen-Sicht sind in diesem Fall alle Inputs ausser links, unten links und unten, gleich null, da keine Schlangenfelder in Sicht sind. Der Input der linken Richtung ist 1, da schon das erste Feld links ein Schlangenfeld ist.

    \columnbreak

    Ausser dem Input von oben links sind alle null, da nur oben links eine Frucht in Sicht ist. Dieser ist dafür eins.

\end{multicols}

\subsubsection{Code Beispiele}

Zuallererst musste ich das Spiel Snake programmieren. Um diesen Prozess zu veranschaulichen, zeige ich am besten die \textCode{action()} Funktion einer Schlange, welche aber noch nicht \enquote{intelligent} ist, und erkläre ihn. Text nach \enquote{\textCode{//}} und zwischen \enquote{\textCode{/*}} und \enquote{\textCode{*/}} ist jeweils ein Kommentar, d.h. er wird nicht ausgeführt:

\lstset{
    language=JavaScript,
    literate=%
    {Ö}{{\"O}}1
    {Ä}{{\"A}}1
    {Ü}{{\"U}}1
    {ß}{{\ss}}1
    {ü}{{\"u}}1
    {ä}{{\"a}}1
    {ö}{{\"o}}1
    {~}{{\textasciitilde}}1
}
\lstinputlisting[inputencoding={utf8}, extendedchars=false, escapeinside={@}{@}]{codes/code1.txt}

Damit man das auch noch bildlich sieht, ist hier ein Resultat der \textCode{this.show(x, y, w, h)} Funktion:

\bigskip
\begin{figure}[h] 
    \begin{center}
        \includegraphics[width=0.4\textwidth]{Downloads/snake.png}
        \caption{\textCode{show(x, y, w, h)} der Klasse \textCode{Snake}}
    \end{center}
\end{figure}

Doch das ist nur eine Schlange, die von uns Menschen gespielt werden kann. Wie sieht es aus, wenn man der Schlange die Kontrolle überlässt? Dafür müssen wir uns die \textCode{think()} Funktion der \textCode{SmartSnake()} Klasse, welche eine Unterklasse der \textCode{Snake} Klasse ist, genauer anschauen. Doch für das müssen wir zuerst noch andere Mechanismen der SmartSnake kennen.

\lstinputlisting[inputencoding={utf8}, extendedchars=false, escapeinside={@}{@}]{codes/code2.txt}

Die Funktion \textCode{predict(input)} eines neuronalen Netzwerkes ist genau das, was im Kapitel "Neuronale Netzwerke" besprochen wird. Die Matrix der Gewichte zwischen Input- und Hidden-Layer wird mit dem Input-Vektor multipliziert, das Ergebnis wird zu den Hidden-Biases addiert und dessen Ergebnis wird von der Activation-Funktion umgerechnet. Dann wird der dabei entstandene Vektor der Länge 8 mit der Matrix der zweiten Gewichte multipliziert, zu der zweiten Bias addiert und nochmal aktiviert. Für die Richtigkeit des Resultats sind natürlich die Gewichte, weights, und die biases. Um zu lernen müssen genau diese auch verändert werden. Dies geschieht in meinem Beispiel dank des genetischen Algorithmus, der die guten Schlangen benützt, um eine neue Generation von Schlangen zu züchten, welche, dank besserer DNA, bzw besserer Netzwerke, durchschnittlich auch bessere Resultate erzielen. Wichtig ist dabei, den guten Schlangen klar höhere Chancen zu geben, aber trotzdem nicht nur mit den besten 5 Schlangen neue züchten, sodass all das wertvolle Erbmaterial im Nu verloren geht und nach wenigen Generationen alle Schlangen identische Netzwerke haben. Dies wird mit sogenannter fitnessproportionaler Selektion geschafft. Diese Selektion ist eine Methode, die eine Schlange mit einer, wie der Name sagt, zu ihrer Fitness proportionalen Chance auswählt, um Kinder machen zu dürfen. Die Fitness ist dabei der Score, von dem wir im Kapitel "genetische Algorithmen" gesprochen haben. Bei meinen Schlangen wird diese Fitness aus der Lebenszeit und der Länge einer Schlange berechnet. Eine Schlange die einen Score von 300 hat, wird also mit doppelter Wahrscheinlichkeit ausgewählt, um ihr Genmaterial an die nächste Generation weiterzugeben, als eine mit dem Score 150. Dieses Verfahren haben wir auch schon bei dein Shakespeare-Affen des Kapitels "genetische Algorithmen" benutzt. Und so sieht 
das in meinem Code aus:

\begin{lstlisting}[
    inputencoding={utf8}, extendedchars=false,  
    escapeinside=``]
class SmartSnake extends Snake {

class Population {
    /*
    Die Klasse Population besteht hauptsächlich aus einer Liste
    von Individuen, in diesem Fall Schlangen, welche Bestand eines 
    genetischen Algorithmus sind. Mit ihr werden alle Prozesse eines 
    GA wie Selektion, Crossover und Mutation ausgeführt.
        this.pop ist die Liste ihrer Individuen.
    */

    ... // andere Funktionen

    fitnessProportionalSelection(count) {
        /* 
        Wenn die Funktion fitnessProportionalSelection benutzt wird, 
        wurde die Fitness aller Schlangen schon berechnet. Der Output
        der Funktion ist eine Liste von selektierten Schlangen 
        der Länge count. In dieser Liste können einzelne (sehr gute) 
        Schlangen auch mehrmals vorkommen.
        */

        // Zuerst wird die Summe aller Fitnessen berechnet
        var sumFit = 0;
        for (var i = 0; i < this.size; i++) {
            sumFit += this.pop[i].fitness;
        }

        // Dann wird die ganze Population in absteigender Reihenfolge 
        // nach der Fitness sortiert.
        this.pop.sort((a, b) => b.fitness - a.fitness);

        /*
        Als nächstes kriegt jede Schlange die neue Variable accFit,
        welche die Summe der Fitnessen aller besseren Schlangen und
        ihrer eigenen Fitness beträgt.
        */
        var accFit = 0;
        for (var i = 0; i < this.size; i++) {
            accFit += this.pop[i].fitness;
            this.pop[i].accFit = accFit;
        }

        /*
        Nun werden die Schlangen ausgewählt, die für das Crossover 
        benutzt werden.
        */
        pool = []; // leere Liste pool (= Mating Pool)
        for (let _ = 0; _ < count; _++) { 
            // Heisst so viel wie: "Folgenden Block count mal wiederholen!".

            // random() = zufällige Zahl zwischen 0 und 1
            rand = random() * sumFit; 
            
            var i = 0;
            // Durch die Population gehen, bis die accFit einer 
            // Schlange grösser als rand ist.
            while (this.pop[i].accFit < rand) { 
                i += 1;
            }
            pool.push(this.pop[i]);
        }
        return pool; // pool hat jetzt die Länge count
    }

    ... // andere Funktionen
}

\end{lstlisting}

\begin{wrapfigure}{r}{0.5\textwidth} 
    \vspace{-20pt}
        \begin{center}
            \includegraphics[width=0.5\textwidth]{Downloads/accFit.png}
            \caption{Darstellung zur Variable \textCode{accFit} und fitnessproportionaler Selektion}
        \end{center}
    \vspace{-15pt}
\end{wrapfigure} 

Der Prozess der Auswahl anhand der Hilfsvariable \textCode{accFit} kann sich auch bildlich vorgestellt werden. Alle Schlangen werden nach der Fitness in absteigender Reihenfolge sortiert. Jede Schlange hat nun einem Kiste mit einer Breite, die gleich ist wie ihre Fitness. Somit entspricht der Abstand des rechten Randes einer bestimmte "Schlangenkiste" zum linken Rand der linksten "Schlangenkiste" der \textCode{accFit} der betroffenen Schlange. Dann wird ein Dart auf die Kisten geworfen, dieser Dart trifft überall, von links bis rechts, mit der gleichen Wahrscheinlichkeit. Die Schlange der getroffenen Schlangenkiste wird ausgewählt. So simpel.

\bigskip
Das Crossover bespreche ich hier nicht, da es schon im Kapitel "Natur- und Computer-Evolution" genau behandelt wurde.

\bigskip
Die Mutation ist wiederum sehr einfach, bei jeder Schlange wird über alle Gene gegangen und bei einer Wahrscheinlichkeit von beispielsweise 1\%, bzw der Mutationsrate, wird das Gen etwas verändert.

\subsubsection{Vergleich der verschiedenen Faktoren} \label{sec:FaktorenVergleich}

Wie ich im Kapitel "genetische Algorithmen" angesprochen habe, gibt es sehr viele Dinge, die die Effizienz eines genetischen Algorithmus beeinflussen können. Die einfachen dieser Faktoren sind die Mutation Rate und die Populationsgrösse. Grosse Populationsgrösse heisst automatisch eine grosse Auswahl an Erbmaterial und somit eine bessere Ausgangslage für die Evolution. Der einzige Nachteil einer sehr grossen Population ist die benötigte Rechenleistung. Denn normalerweise möchte man ja einen effizienten Algorithmus im Bezug auf die Zeit, nicht die Generationen. Doch unser Ziel ist weniger die Zeiteffizienz als die Generationeneffizienz, da es um die evolutionären Prozesse geht, die wir vergleichen wollen und diese in der echten Welt wichtiger ist. Ausserdem ist die Ziet, die es braucht Zufälle und Scores auszurechnen, in der echten Welt gleich null. Mutationrate ist etwas weniger eindimensional. Wird sie zu klein, tendiert die Population zur Kongruenz, also der Homogenisierung aller Individuen. Ist sie jedoch zu gross, wird jeglicher Fortschritt der natürlichen Selektion rückgängig gemacht, da zu viel DNA mutiert. Dann gibt es noch die Fitness-Funktion, die einen sehr grossen Einfluss auf die Evolution haben kann. Denn wenn die Funktion zu flach ist, dann erfüllt die fitnessproportionale Selektion ihren Sinn nicht mehr, da die besseren Schlangen kaum eine höhere Chance als der Rest haben. Ist sie wiederum zu exponentiell, kann es schnell vorkommen, dass eine ganze neue Generation nur aus einer handvoll Schlangen gezüchtet. Deshalb ist es ein ziemlich schwieriger Balanceakt, eine gute Fitness-Funktion für eine komplexe Sache wie die Leistung einer Schlange zu finden. Letztens spielt in meinem Projekt der Aufbau des neuronalen Netzwerkes eine grosse Rolle. Es kann zu wenig komplex sein, um die Situation angemessen zu einer Richtung zuzuteilen oder auch so gross, dass der Output fast nicht mehr vom ursprünglichen Input abhängt. Ausserdem führen grössere neuronale Netzwerke auch schneller zu Speicherproblemen. Ich vergleiche nun alle diese Faktoren, indem ich für jede Art Einstellung den genetischen Algorithmus 15 Generationen lasse.

\begin{center}
    \includegraphics[width=1\textwidth]{Downloads/bigbigrun.png}
\end{center}

Zuerst erkläre ich alle diese Einstellungen:
\begin{itemize}  
    \item Netzwerke: 
    \begin{enumerate}
        \item Net 1 (Spalten 1 - 3): Ein Netzwerk der Struktur (24, 8, 4) (24 Inputs, 8 Neuronen der Hidden Layer und 4 Ouputs)
        \item Net 2 (Spalten 4 - 6): Ein Netzwerk der Struktur (24, 8, 8, 4)
        \item Net 3 (Spalten 7 - 9): Ein Netzwerk der Struktur (24, 16, 4)
        
    \end{enumerate}
    \item Mutation Rates:
    \begin{enumerate}
        \item 0.002mr (Spalten 1, 4, 7): Mutation Rate = 0.2 Prozent
        \item 0.01mr (Spalten 2, 5, 8): Mutation Rate = 1 Prozent
        \item 0.05mr (Spalten 3, 6, 9): Mutation Rate = 5 Prozent 
    \end{enumerate}
    \item Populationsgrössen:
    \begin{enumerate}
        \item 1000P (Zeilen 1 - 5): Grösse der Population = 1000
        \item 5000P (Zeilen 6 - 10): Grösse der Population = 5000
        \item 10000P (Zeilen 11 - 15): Grösse der Population = 10000
    \end{enumerate}
    \item Fitness-Funktionen:
    \begin{enumerate}       
        \item 1cf (Zeilen 1, 6, 11): erste Fitness-Funktion (mehr später)
        \item 2cf (Zeilen 2, 7, 12): zweite Fitness-Funktion 
        \item 3cf (Zeilen 3, 8, 13): dritte Fitness-Funktion
        \item 4cf (Zeilen 4, 9, 14): vierte Fitness-Funktion
        \item 5cf (Zeilen 5, 10, 15): fünfte Fitness-Funktion
    \end{enumerate}
\end{itemize}

Am Auffälligsten sind wahrscheinlich die verschiedenfarbigen und verschieden satten Hintergründe der Diagramme. Doch zuerst muss ich die eigentlichen Diagramme erklären, damit diese ebenfalls Sinn ergeben. Die X-Achse sind jeweils die 15 Generationen eines Versuches. Die Y-Achse hat die Einheit des Scores der Fitness-Funktion. Die blaue Linie zeigt jeweils die besten Scores einer einzelnen Generation. Die orange den Durchschnitt der Generation und die grüne den schlechtesten Score. Dementsprechend zeigt ein Diagramm den Verlauf der Leistungen einer Population über die Generationen. Diese Leistung wird anhand der Besten, des Schnitts sowie den Schlechtesten aufgezeigt. Die Farben dienen der Unterscheidung der Fitness-Funktionen, denn ein Score von 200 bei der zweiten Funktion ist nicht das gleiche wie ein Score von 200 bei der fünften Fitness-Funktion. Die Sättigkeit des Hintergründe zeigt das Verhältnis zwischen eigenem Maximalscore und dem Maximalscore aller Versüche dieser Fitness-Funktion. Die rötesten Diagramme sind also die Versüche mit den höchsten Scores aller Versüche auf Zeilen 1, 6 und 11.

\bigskip
Doch nun zur Analyse, gut ersichtlich ist, dass die mittleren 3 Spalten meistens weniger satt sind, als links und rechts. Folglich ist das Net 2, also die Struktur (24, 8, 8, 4) nicht so gut wie die beiden Anderen. Die Unterschiede zwischen denn verschiedenen Netzwerk-Strukturen sind mit einem etwas anderen Test noch besser ersichtlich. Ich lasse eine riesige Population von 80000 Schlangen zufällig initialisieren und das Game spielen und schaue dann direkt die Resultate an, ohne Evolution stattfinden zu lassen. Somit ist das Glück, dass beim Net 3 vielleicht einfach eine bessere Anfangspopulation vorhanden war, auszuschliessen.

\begin{center}
    %\includegraphics[width=0.47\textwidth]{Downloads/barplotA.png}
    \includegraphics[width=0.47\textwidth]{Downloads/barplotB.png}
    \includegraphics[width=0.47\textwidth]{Downloads/barplotMW.png}
\end{center}

Da die besten Fitnessen der Population so viel höher sind als der Schnitt, ist es besser die besten gesondert zu betrachten. Nun ist definitv zu erkennen, dass das Net 3 die anderen übertrifft, denn nicht nur der beste Score sondern auch der Schnitt ist dort jeweils am höchsten.

\bigskip
Um die Verschiedenheiten der verschiedenen Fitness-Funktionen zu verstehen, müssen wir uns zuerst die Funktionen anschauen:\\
pow(x, y) ist $x^y$.

\lstset{
  language=Python,
  aboveskip=3mm,
  belowskip=3mm,
  showstringspaces=true,
  columns=flexible,
  basicstyle={\small\ttfamily},
  numbers=none,
  numberstyle=\tiny\color{gray},
  keywordstyle=\color{blue},
  commentstyle=\color{dkgreen},
  stringstyle=\color{mauve}
}

% Python Fitness Function
\begin{addmargin}[2em]{0em}
\begin{lstlisting}
    def cf1(lifetime, length):
    if length < 10:
        return lifetime*lifetime*pow(2, length)/1000
    else:
        return lifetime*lifetime*pow(2,10)*(length-9)/1000
def cf2(lifetime, length):
    if length < 10:
        return lifetime * 0.5 + length * length
    else:
        return 9 * 9 + (length - 9) * 8 + 100 + (lifetime - 100) * 0.5
def cf3(lifetime, length):
    if length < 10:
        res = lifetime * 0.5 + length * length
    else:
        res = lifetime * 0.5 + 81 + (length-9) * 10
    return res * res
def cf4(lifetime, length):
    return lifetime * 0.1 + length * length

def cf5(lifetime, length):
    return (lifetime * lifetime * 0.1 * pow(2, length))
\end{lstlisting}
\end{addmargin}

Wenn man das visualisiert, sieht das so aus:

\begin{center}
    \includegraphics[width=0.325\textwidth]{Downloads/cf1.png}
    \includegraphics[width=0.325\textwidth]{Downloads/cf2.png}
    \includegraphics[width=0.325\textwidth]{Downloads/cf3.png}
\end{center}


\begin{center}
    \includegraphics[width=0.325\textwidth]{Downloads/cf4.png}
    \includegraphics[width=0.325\textwidth]{Downloads/cf5.png}
\end{center}

Die X-Achse ist jeweils die Länge der Schlange, die Y-Achse die Lebenszeit und die Z-Achse die Fitness.

\bigskip
Die zweite und vierte Fitness-Funktion scheinen ziemlich flach zu sein. Dies sieht man auch bei den Evolutionsresultaten. Erstens ist im Laufe der Generationen kein Fortschritt erkennbar, was auf die mangelnde Überlegenheit der besseren Schlangen zurückzuführen ist. Dh die besseren Schlangen, die eigentlich ihr Erbmaterial vererben sollten, können es nicht genug vererben, da die Selektion sie kaum bevorzugt. Ausserdem führen alle Versuche zu sehr ähnlichen Resultaten.

\bigskip
Die fünfte Fitness-Funktion ist definitv zu exponentiell, da bei ihr jeweils vereinzelte Versuche (durch Glück) einen im Vergleich wahnsinnig hohen Score zustande bringen und dieser mit gleicher DNA, die offensichtlich vererbt worden sein muss, oft nicht nochmals möglich ist.

\bigskip
Das Problem der dritten Funktion ist hauptsächlich, dass es die Lebenszeit zu gewichtig zählt. Das führt zu Schlangen, die endlos im Kreis drehen. Und da ich für Trainingszwecke Schlangen nach einer zu langen Zeit, keine Frucht zu essen, sterben lasse, führt das zu keinen guten Resultaten. Man fragt sich dann vielleicht, wieso ich überhaupt die Lebenszeit in die Fitness miteinbeziehe, weil doch eigentlich nur die Länge zählt. Wenn ich sie nicht mit einbeziehen würde, würden die anfangs so schlechten, zufällig initialisierten Schlangen alle eine Fitness von null haben. Wenn aber doch einige eine Frucht gegessen haben, ist das zu einem grossen Teil pures Glück, weil die Frucht gerade auf dem Weg lag.

\bigskip
Nun bleibt noch die erste, bei ihr ist die Steigung grundsätzlich schon ziemlich passend. Wenn man die Evolutionsresultate anschaut, wird das auch bestätigt. Die Schlangen verbessern sich richtig, da die blauen Graphen oft kontinuierlich steigen. Das ist bei den anderen Fitness-Funktionen oft nicht der Fall. Ausserdem hat es diese vereinzelten Spitzen nicht ganz so oft wie etwa Funktion 3 und 5. Diese Spitzen würden darauf hindeuten, dass die Funktion zu exponentiell ist. Das scheint sie nicht zu sein.

\bigskip
Bei den Mutationsraten sieht man vorallem in den letzten drei Spalten, wie die 5\%ige die besten Resultate bringt. Doch in den Vergleichen werden auch relativ kleine Populationsgrössen benutzt. Deshalb werde ich beim Endexperiment trotzdem eine Mutationsrate von einem Prozent benutzen, da ich dann auch grössere Populationen benutzen werde und somit trotzdem genug Mutation vorhanden ist.

\bigskip
Offensichtlich ist eine Population von nur 1000 Schlangen zu klein, da dort klar die schlechtesten Resultate erzielt wurden. Zwischen 5000 und 10000 sieht man zwar fast keinen Unterschied, aber ich werde sowieso noch etwas mehr benutzen, da ich dann im Vergleich zwischen Natur-Evolution und Computer-Evolution nur zweimal Evolution laufen lassen muss und nicht wie in diesem Quervergleich 135 Mal für die 5 verschiedenen Fitness-Funktionen, 3 Populationsgrössen, 3 Mutationsraten und 3 Netzwerkstrukturen.

\bigskip
Zusammenfassend entscheide ich mich also für eine 1\%ige Mutationsrate, die erste Fitness-Funktion, das Net 3 und eine 25000 grosse Population. 25000 nehme ich schlicht, weil es das Maximum ist, das noch keine Speicherprobleme mit sich bringt. Mit diesen Einstellungen denke ich, dass der Vergleich zwischen Natur-Evolution und Computer-Evolution gute Konditionen hat und zu den am besten brauchbaren Resultaten führen wird.

\subsection{Vergleich Natur- und Computer-Evolution}

In diesem Kapitel komme ich zum Endziel dieses Teils meiner Maturaarbeit. Der Vergleich zwischen der Natur-Evolution und der Computer-Evolution. Für das lasse ich den Genetischen Algorithmus zweimal laufen. Einmal haben die Schlangen die DNA-Architektur, die ich für die Natur-Evolution definiert habe, mit den doppelten Chromosomen und einem Crossover, das analog zur echten Welt funktioniert. Beim zweiten Mal ist dann alles wie bei einem herkömmlichen genetischen Algorithmus, also der Computer-Evolution. 

\begin{center}
    \includegraphics[width=1\textwidth]{Downloads/finalA.png}
    \includegraphics[width=1\textwidth]{Downloads/finalMW.png}
\end{center}

Sehr spannende Resultate, welche genaue Auseinandersetzung benötigen, um verstanden zu werden. Zuerst rein objektive Beobachtungen: Die maximale Fitness, die die Computer-Evolution erreichte, ist 338 Millionen während die Natur-Evolution auf höchstens 5.1 Millionen kam. Die Computer-Evolution in Generation 7 66 Prozent der maximalen Fitness und die Natur-Evolution in Generation 10 82 Prozent ihres Rekords. Natur- wie auch Computer-Evolution steigerten ihre besten Scores nach den ersten Spitzen nur noch wenig. Der Durchschnitt erreichte bei der Computer-Evolution den Spitzenwert von knapp 2.5 Millionen in Generation 25. Schon in der dreizehnten Generation gipfelte der Durchschnitt der Natur-Evolution bei einer Fitness von 1026. Der Schnitt der Computer-Evolution verschlechterte sich kein einziges Mal. Die Natur-Evolution verschlechterte sich im Schnitt einige Male und machte ab der zehnten Generation mehr oder weniger keinen Fortschritt.

\bigskip
Offensichtlich hat die Natur-Evolution einiges schlechter abgeschnitten. Doch anhand weiterer Daten habe ich bemerkt, dass bei der Computer-Evolution etwas aussergewöhnliches passiert ist. In der nullten Generation, bzw. die die zufällig initialisiert wurde, gab es eine Schlange die für die zufällige Initialisierung wahnsinnig gut abgeschnitten hat. Sie hatte eine Fitness 412000, was extrem hoch ist. Ganz allein hat sie den Durchschnitt von 8.4 auf knapp 25 angehoben. Somit entstand eine seltsame erste gezüchtete Generation. 11000 Kopien der besten Schlange entstanden, da als Vater wie auch als Mutter diese ausgewählt wurde, ja das ist hier möglich. Dazu ist bei weiteren 11000 Schlangen nur ein Elternteil die gute, was nur noch 3000 Schlangen übrig lässt, die nicht von der Mega-Schlange abstammen. Spätestens in der zweiten und dritten gezüchteten Generation wird dieses \enquote{Geschlecht} die ganze Population übernommen haben. Deshalb habe ich das Experiment wiederholt. Da ich in meiner Hypothese an die langfristige Überlegenheit der Natur-Evolution glaubte, habe ich diese nun doppelt solange laufen lassen. Ziemlich ähnliche Resultate sind dabei herausgekommen:

\begin{center}
    \includegraphics[width=1\textwidth]{Downloads/final2A.png}
    \includegraphics[width=1\textwidth]{Downloads/final2MW.png}
\end{center}

Zuerst wieder zu den Zahlen: Dieses Mal erreichte die Computer-Evolution eine maximale Fitness von fast 180 Millionen, der Rekord der Natur-Evolution ist 16300. Im Gegensatz zum ersten Durchlauf sieht man jetzt besser die stetige Steigerung der Topscores der Computer-Evolution. Bei der Natur-Evolution ist die Maximal-Fitness wieder ab 10. Generation stehen geblieben, und zwar bei klar tieferen Resultaten als im ersten Durchlauf. Den höchsten Durchschnitt erreichte die Natur-Evolution diesmal eher am Schluss, nämlich in der 41. Generation, sie hatte in der 10. Generation jedoch schon 95\% dessen und blieb danach wie gesagt stehen. Die Computer-Evolution hingegen hat sich wie beim ersten Mal nicht einmal im Schnitt verschlechtert, vielmehr hat sie sich stetig zu höheren Durchschnitten emporgearbeitet.

\bigskip
Meine Hypothese war, dass die Natur-Evolution nach anfänglicher Führung der Computer-Evolution schlussendlich bessere Resultate erreichte. Diese lag offensichtlich ziemlich gänzlich falsch, da ich das Potenzial der Computer-Evolution ziemlich unterschätzt habe und ich vielleicht zu viel Hoffnung auf unsere Natu gesetzt habe. Im Folgenden werde ich die unerwarteten Resultate erläutern.

\bigskip
Die Natur-Evolution hat doch einige Nachteile, denen ich mir anfangs nicht bewusst war. Zum Beispiel ist die dominante Erbinformation weniger divers. Wegen der Dominanz der stärkeren Genen (grössere Zahl) ist die Wahrscheinlichkeit, dass ein Gen grösser als 0.5 ist, nicht 50\%, sondern 75\%. Dies beeinflusst die Effizienz der neuronalen Netzwerke, denn wenn alle Gewichte und Biases ähnlicher sind, funktioniert der Denkprozess nicht so gut. Man braucht ja genau ziemlich verschiedene Gewichte um die Inputs unterschiedlich zu behandeln. Das kann man sich auch etwa so vorstellen wie der Unterschied zwischen einer schwarz-weissen Sicht und farbiger Sicht, was aber eher überspitzt ist.

\bigskip
Andererseits ist die Natur-Evolution auch insofern im Nachteil, wie die Leistung einer Schlange gar nicht unbedingt das symbolisiert, was sie vererben kann. Es kann also sein das sich die dominanten Gene des Chromosomenpaars einer bestimmten Schlange perfekt kombinieren. Wenn es nun Kinder macht, vererbt sie jedoch jeweils nur eines ihrer beiden Chromosome. Dieses könnte zusammen mit dem anderen Chromosom, welches das Kind erbt, ganz andere dominante Gene hervorbringen, die gar nicht gut funktionieren. Die Computer-Evolution ist natürlich auch davon abhängig, ob eine gute Paarung gemacht wird, die ein zufriedenstellendes Kind zeugt. Jedoch vererbt eine Schlange, die in der Computer-Evolution eine hohe Fitness hatte, mit Sicherheit auch die Gene, welche sie selbst benutzte. Dies führt zu einer grösseren Verlässlichkeit, das gute Schlangen auch wirklich guten Nachwuchs zeugen.

\bigskip
Im Vorhinein sagte ich, dass die Natur-Evolution von einer doppelt so grossen Menge Erbmaterial profitieren könnte. Doch offensichtlich geht es mehr um die richtige Kombination als, um die richtige Zahl bei einem bestimmten Gen. Die Natur-Evolution hat, was Kombinationspotenzial angeht, einen sehr grossen Nachteil. Wenn man  sich die Mutation wegdenkt, hat die Natur-Evolution pro Chromosom genau doppelt so viele Exemplare, wie Schlangen in der Population sind. Weil es sich bei diesem Projekt, um eine DNA mit nur einem einzigen Chromosom, dem neuronalen Netzwerk, handelt, hat es bei 25000 Schlangen, 50000 Chromosome mit der Information für je ein Netzwerk. Ein einzelnes Individuum hat jeweils zwei solche Exemplare als Chromosomenpaar, woraus sie die dominanten Gene wählt. Somit sind eigentlich $ {50000 \choose 2} = 1249975000 $ verschiedene Chromosomenpaare und somit Schlangen. Vernachlässigt ist hierbei wie gesagt Mutation und der Verlust grosser Teile an Information über die Generationen, sodass nie alle 1.2 Milliarden Chromosomenpaare realisiert werden. Die Anzahl dieser Möglichkeiten scheint vielleicht gross zu sein, doch wenn man sich die Computer-Evolution anschaut, sieht das ganz anders aus. Für jedes einzelnes Gen gibt es so viel Exemplare wie Schlangen. Dh eigentlich halb so viele wie bei der Natur-Evolution. Doch da bei der Computer-Evolution immer wieder neu Chromosome aus den Elternchromosomen gemischt werden und diese auch vererbt werden (Bei der Natur-Evolution wird auch aus Vater- und Mutterchromosom gemischt, doch sozusagen die Hälfte der Zutaten vererbt, anstatt dem Gemisch), gibt es für jedes Gen 25000 Varianten, die beliebig kombiniert werden können. Weil das Neuronale Netzwerk 468 Parameter, also alle Gewichte und Biases hat, gibt es somit $ 25000^{468} $ verschiedene Schlangen, die ohne jegliche Mutation gezüchtet werden könnten. Das ist $ 1.72e2058 $. Die Natur-Evolution hatte $ 1.25e9 $. Das ein riesiger Unterschied. Die Unabhängigkeit von den Chromosomen bringt diesen entscheidenden Vorteil. In meinem hatten die Schlangen zwar auch nur ein einziges Chromosom, was den Nachteil der Natur-Evolution noch verstärkt, doch in unserer Natur haben die meisten Lebewesen unter 100 Chromosomen und vereinzelt bis zu 1300 \footnote{(http://www.biologie-lexikon.de/lexikon/chromosomenzahl.php)}. Das ist zwar klar mehr als in meinem Projekt, aber es sind natürlich auch viel mehr Gene vorhanden, beispielsweise beim Menschen zählt man zwischen 25 bis über 120 Tausend. Nur schon mit 20000 Genen, übersteigt das die Anzahl Chromosome genug, um bei der Computer-Evolution ein Vielfaches mehr verschiedene Kombinationen zu ermöglichen, als sich bei der Natur-Evolution je möglich ist.

\pagebreak
\section{Fingeralphabet-Erkennung}

\subsection{Idee und Ziele}

Mein zweites grosses Projekt, besteht daraus ein Endprodukt zu erarbeiten, welches auf künstlicher Intelligenz basiert. Ich werde ein sogenanntes Convolutional Neural Network, eine spezielle Form eines neuronalen Netzwerks, programmieren und dieses trainieren, Fotos von Händen in das Fingeralphabet der Deutschschweiz zu kategorisieren. Das Fingeralphabet ist ein Teil der Gebärdensprache und hat für jeden Buchstaben ein Zeichen, sowie für "SCH" und "CH". Mein Ziel ist es also eine künstliche Intelligenz zu erschaffen, die das Fingeralphabet versteht. Das ist ein sogenanntes Kategorisierungsproblem. Die Fotos müssen in 29 Kategorien eingeteilt werden, 26 Buchstaben, "SCH", "CH" und Nichts. Denn die KI muss ja auch wissen, wann gar kein Zeichen gezeigt wird. In diesem Teil, fokusiere ich mich weniger darauf alles für Laien verständlich zu machen, sondern benutze manchmal Fachbegriffe, ohne sie weiter zu erklären, und erwarte somit hin und wieder Vorwissen. Doch die groben Grundzüge werden auch interessierte Laien verstanden haben und wichtig ist vorallem, dass das Endprodukt funktioniert. Denn das ist dann benutzbar ohne jegliches Wissen über Machine Learning. Das Datenset, mit welchem meine künstliche Intelligenz trainiert wird, mache ich selbst. Ich drehe Videos von verschiedenen Händen, die jeweils solche Zeichen formen, mit verschiedenen Hintergründen. Ich berufe mich dabei auf die Vorlage von Sonos, dem schweizerischen Hörbehindertenverband. Die Videos teile ich dann in einzelne Fotos auf. Ich werde am Schluss von jeder Kategorie 3000 Bilder haben. Dh insgesamt 87000 Fotos mit dem Format 320x240. Das ist ein ziemlich kleines Format, aber es reicht um die Hände zu erkennen und grössere Formate würden auch mehr Rechenleistung verlangen. Ich denke, durch das Training einer solchen KI kann man ein sehr praktisches Endprodukt von grossem Nutzen erschaffen. Das Fingeralphabet wird zwar grundsätzlich nur benutzt, um zu buchstabieren und Wörter ohne eigenes Gebärdenzeichen auszudrücken, aber grundsätzlich sind einem Alphabet ja keine Grenzen gesetzt. Ausserdem wäre es ein Projekt einer ganz anderen Dimension, wenn man möchte, dass eine KI alle oder sehr viele Gebärdenzeichen erkennen kann. 

\subsection{Wie lernt ein neuronales Netzwerk?}

Im ersten Teil meiner Maturaarbeit habe ich einen genetischen Algorithmus benutzt, um den Lerneffekt zu erzeugen. Dabei habe ich einfach gute neuronale Netzwerke rausgepickt und neu zusammengestellt, doch nie hat sich eine einzelne Schlange in ihrer Lebenszeit verbessert, alles ging jeweils über die nächste Generation. In den meisten Bereichen von Machine Learning wird das nicht so gehandhabt. Normalerweise werden verschiedene Formen des Gradient Descent benutzt. Gradient Descent ist eine Methode, die für jeden einzelnen Parameter, also die einzelnen Gewichte und Biases, eine Ableitung ausrechnet. Diese Ableitung ist das Verhältnis des jeweiligen Parameters und des Fehlers des Netzwerkes. Und wenn man dieses Verhältnis kennt, kann man den Parameter in eine Richtung ändern, in welcher der Fehler, also der Unterschied zwischen dem Output des Netzwerks und des erwünschten Outputs, kleiner wird. Wenn man diesen Gradient Descent, also das Umherschieben der Parameter, sehr oft wiederholt, nimmt das neuronale Netzwerk langsam die erwünschte Form an. Und so werde ich auch den nächsten Teil meiner Maturaarbeit lernen lassen. Ich werde jedoch nicht auf die Details eingehen, da man sich mit diesen als Laie sehr intensiver auseinandersetzen müsste, um es zu verstehen.

\subsection{CNN}

Convolutional Neural Networks werden hauptsächlich benutzt, um Inputs in Form von Bildern zu verarbeiten, können aber auch an Audio-Inputs und eindimensionalen Inputs angewendet werden. Sie funktionieren eingermassen gleich wie traditionelle Neural Networks. Man gibt ihnen einen gewissen Input und sie geben wiederum Outputs einer bestimmten Anzahl zurück. Sie haben auch solche Layers wie ich beim Kapitel "Neuronale Netzwerke" zeigte, doch die ersten paar Layers sind speziell. Es sind sogenannte Convolutional Layers, zu Deutsch etwa "faltende Schicht". Convolutional Layers sind sozusagen Filter die über ein Bild geschoben werden. Beziehungsweise besteht eine solche Layer meist aus mehreren Filtern und produziert somit mehrere geefilterte Versionen des Bildes. Ein einzelner Filter hat eine bestimmte Breite und Höhe und besteht aus einer Matrix aus Gewichten. Wenn nun ein Filter, oder auch Kernel, über das Bild gehalten wird, wird jeweils der Pixelwert mit dem Gewicht, das darüber liegt, multipliziert. All diese Produkte werden summiert und das ist dann der Wert des einen Pixel des "gefilterten" Bildes. An dem Resultat einer Convolutional Layer wird wieder eine Activation Function angewendet, analog zu normalen Neural Networks. Dann werden meist auch noch sogenannte Maxpooling Layers benutzt, welche eigentlich lediglich das Resultat zu kleineren Formaten skalieren. Diese 3 Schritte werden einige Male wiederholt. Dann werden die Pixelwerte des Resultat zu einer langen Kette aufgereiht. Diese Kette von Zahlen werden dann von sogenannten Fully Connected Layers, den ganz normalen, die wir im Kapitel "Neuronale Netzwerke" angeschaut haben, weiter verarbeitet. Also muss man sich das so vorstellen, dass ein Input-Bild durch ganz viele spezielle Filter geschickt werden und danach ein normales neuronales Netzwerk folgt.

\subsection{Mein Projekt}

Für mein Bilderkennungsprojekt habe ich zuerst drei phD Studenten der ETH kontaktiert. Denn anfangs war noch unklar, was ich für ein Machine Learning Projekt machen möchte. Ich habe auf der Website Kaggle\footnote{www.kaggle.com} viele verschiedene Datensets durchsucht. Irgendwann stiess ich auf ein Datenset mit Bildern der American Sign Language. Eine bildliche Sprache in das lateinische Alphabet umwandeln zu können, fand ich eine tolle Vorstellung. Darum habe ich mich dann für das entschieden. Die drei Studenten haben mir dann sehr wichtige Tipps zur Strukturierung des Arbeitsablaufs gegeben. 

\subsubsection{Schwarz-Weiss-Buchstaben}

Zuerst machte ich einen Probelauf mit Bildern von lateinischen Buchstaben, welche zufällig gross und zufällig gedreht sind. Diese Bilder sind sehr klein mit einem Format von 32x32 Pixeln. Ich habe die Bilder selbst generiert mit einem simplen Programm. Das sieht beispielsweise so aus:

\begin{center}
    \includegraphics[width=1\textwidth]{Downloads/bwgen.png}
\end{center}

Dann programmierte ich mit der Machine Learning Bibliothek \enquote{pytorch}, der Tipp der ETH-Studenten, ein erstes Convolutional Neural Network. Da die Bilder sehr klein waren, war die gebrauchte Rechenleistung gering und ich konnte es sogar auf meinem eigenen Laptop trainieren. Ich habe pro Buchstaben 2000 Trainings-Bilder generiert und dann noch je 200 für Testzwecke. Denn sehr wichtig beim Training von künstlichen Intelligenzen ist, dass man noch einige Bilder vom Trainingsset separiert. Somit ist das Testen viel realitätsgetreuer, da sie auf unbekannte Bilder getestet werden. Ausserdem könnten die Neuronalen Netzwerke nach zu langem Training "zu gut" sein, auf englisch sagt man "overfitting". Dh sie können die Bilder aus dem Trainingsset so gut, dass sie die sozusagen auswendig können und nicht mehr auf andere Bilder anwendbar sind. Deshalb behält man immer einen Teil des Datensets beiseite, um sie besser testen zu können. Nun zu den Resultaten des CNNs, welches die Schwarz-Weiss-Buchstaben kategorisiert:

\begin{center}
    \includegraphics[width=0.45\textwidth]{Downloads/BWresults/ep0.png}
    \includegraphics[width=0.45\textwidth]{Downloads/BWresults/ep1.png}
    \includegraphics[width=0.45\textwidth]{Downloads/BWresults/ep2.png}
    \includegraphics[width=0.45\textwidth]{Downloads/BWresults/ep3.png}
    \includegraphics[width=0.45\textwidth]{Downloads/BWresults/ep4.png}
    \includegraphics[width=0.45\textwidth]{Downloads/BWresults/ep5.png}
    \includegraphics[width=0.45\textwidth]{Downloads/BWresults/ep10.png}
    \includegraphics[width=0.45\textwidth]{Downloads/BWresults/ep15.png}
\end{center}

Auf der Zeile ist jeweils der Buchstabe, der getestet wurde, d.h. das Ziel für das Netzwerk. Die Summe einer Zeile ist immer Hundert. Denn in einer Zelle steht, wie viel Prozent der getesteten Bilder des Buchstabens der Zeile mit dem Buchstaben der Spalte beantwortet wurde.

\bigskip
Eine Epoche ist der Prozess einmal jedes Bild des Trainingsset zu lernen.

\bigskip
Wie man gut sehen kann, hat das anfangs zufällig initialisierte Neuronale Netzwerk eine Vorliebe für einige der Buchstaben. Egal, was getestet wird, antwortet es sehr häufig mit \enquote{G}, da es einfach so \enquote{geboren} wurde. Doch während des Trainings wird es langsam verbessert und lernt verschiedene Antworten zu geben. Dabei sieht man sehr schön, wie sich die Farbe langsam bei der Diagonalen festigt, weil mehr und mehr die Tests pro Buchstabe 100\%ig korrekt beantwortet werden. Von Epoche 10 bis 100 steigert sich die Präzision nur noch von 98.3\% zu 99.8\%. Das ist eher schlecht als recht. Denn wenn ein neuronales Netzwerk seine Trainingsdaten zu gut kennt, führt das zu dem vorhin genannten Overfitting. Dh es kennt die Bilder des Trainings sozusagen auswendig und ist nicht mehr auf ihm unbekannte Bilder anwendbar. Dies ist möglichst zu vermeiden.

\subsubsection{Fingeralphabet}

Nach einem erfolgreichen Testlauf mit den schwarzweissen Buchstaben, konnte ich mich der wahren Aufgabe zuwenden. Ursprünglich habe ich ein Datenset der American Sign Language auf Kaggle\footnote{https://www.kaggle.com/grassknoted/asl-alphabet} gefunden. Dieses Datenset war jedoch sehr gleichmässig, d.h. die Handzeichen wurden stets vor exakt gleichem Hintergrund aufgenommen, nur nicht immer in der Mitte. Deshalb wollte ich selbst ein Datenset erstellen. Für das machte ich Videos des deutschschweizer Fingeralphabets mit meiner Familie. Ich benutzte also drei verschiedene Hände und drehte jeweils die Kamera beim Filmen, sodass sich der Hintergrund stetig etwas verändert. Ausserdem drehte ich an fünf Stellen mit unterschiedlichen Hintergründen. Würde man nur einmal durch das Alphabet gehen, kann ein sehr ungewollter Fehler passieren. Da jeder Buchstabe einige Minuten später gefilmt wird, ist das Sonnenlicht immer etwas schwächer und somit könnte die künstliche Intelligenz die Buchstaben einfach anhand der Helligkeit des Hintergrunds erkennen. Deshalb bin ich auch mehrmals und zu verschiedenen Tageszeiten durch das Alphabet gegangen und habe alle Zeichen gefilmt. Doch eigentlich wollte ich ja Fotos der Handzeichen. Da ich 3000 Bilder jedes Zeichens brauchte, war einzeln fotografieren eigentlich keine Option. Vielmehr habe ich jedes hundert Sekunden lang bei dreissig Bildern pro Sekunde gefilmt und die Resultate mit einem Python-Skript in Bilder des JPEG-Formats umgewandelt. Somit hatte ich mit verhältnismässig wenig Aufwand schnell mein Datenset. Mit 26 Buchstaben, einem SCH, einem CH und einer Kategorie, die nichts symbolisiert, hatte ich $ 29*3000 = 87000 $ Bilder. Nun konnte ich mich dem Trainieren des CNN widmen.

\bigskip
Zuerst habe einige Änderungen bei der Struktur des Netzwerks vorgenommen, da das Format der Bilder nun 320x240, und nicht mehr 32x32, ist und jetzt in 29 Kategorien anstatt in 27 zu kategorisieren ist. Nach einer ersten Version, die mit dem neuen Datenset funktionierte, habe ich zuerst einfach einen Testlauf gemacht, um zu schauen wie das Netzwerk sich so macht. Jedoch gibt es bei diesem Datenset ein neues Problem, die Rechenleistung meines Laptops reichte lang nicht mehr aus. In einer ganzen Nacht kam er mit dem Training nicht annähernd so weit wie erhofft. Doch die ETH-Studenten wussten natürlich weiter. Ich konnte mich mit SSH und SCP über einen Server mit einem Computer mit einem GPU, einer Grafikkarte, ihrer Abteilung verbinden und von meinem Laptop aus steuern. Das ermöglichte mir auf eine viel höhere Rechenleistung zuzugreifen. 

\subsection{Schluss}

\section{Outro}

\end{document}